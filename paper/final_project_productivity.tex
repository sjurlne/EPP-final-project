\documentclass[11pt, a4paper, leqno]{article}
\usepackage{a4wide}
\usepackage[T1]{fontenc}
\usepackage[utf8]{inputenc}
\usepackage{float, afterpage, rotating, graphicx}
\usepackage{epstopdf}
\usepackage{longtable, booktabs, tabularx}
\usepackage{fancyvrb, moreverb, relsize}
\usepackage{eurosym, calc}
% \usepackage{chngcntr}
\usepackage{amsmath, amssymb, amsfonts, amsthm, bm}
\usepackage{caption}
\usepackage{mdwlist}
\usepackage{xfrac}
\usepackage{setspace}
\usepackage[dvipsnames]{xcolor}
\usepackage{subcaption}
\usepackage{minibox}
% \usepackage{pdf14} % Enable for Manuscriptcentral -- can't handle pdf 1.5
% \usepackage{endfloat} % Enable to move tables / figures to the end. Useful for some
% submissions.

\usepackage[
    natbib=true,
    bibencoding=inputenc,
    bibstyle=authoryear-ibid,
    citestyle=authoryear-comp,
    maxcitenames=3,
    maxbibnames=10,
    useprefix=false,
    sortcites=true,
    backend=biber
]{biblatex}
\AtBeginDocument{\toggletrue{blx@useprefix}}
\AtBeginBibliography{\togglefalse{blx@useprefix}}
\setlength{\bibitemsep}{1.5ex}
\addbibresource{../../paper/refs.bib}

\usepackage[unicode=true]{hyperref}
\hypersetup{
    colorlinks=true,
    linkcolor=black,
    anchorcolor=black,
    citecolor=NavyBlue,
    filecolor=black,
    menucolor=black,
    runcolor=black,
    urlcolor=NavyBlue
}


\widowpenalty=10000
\clubpenalty=10000

\setlength{\parskip}{1ex}
\setlength{\parindent}{0ex}
\setstretch{1.5}


\begin{document}

\title{Final Project: Productivity Analysis Across Sectors}

\author{Sjur Løne Nilsen}

\date{
    {\bf Universität Bonn}
    \\[1ex]
    \today
}

\maketitle


\begin{abstract}
    This paper presents a productivity analysis framework for various sectors in a country's 
    economy using web scraping techniques. The analysis focuses on Labour Productivity and 
    Total Factor Productivity. The productivity measure standards of the Statistical Bureau 
    of Norway are used as a reference. The paper presents changes in productivity over time 
    and levelled graphs for the 10 largest sectors. It also compares the paths of five large
    sectors in Norway, Denmark and Sweden. The resulting framework can provide 
    insights into productivity and is espiacially aimed at policy consultants, where much of
    the calculation and collection of data is automatized. As this paper is written in the winter 
    2022/2023 the use of ChatGPT is reported at the end, as it could potentially provide valuable 
    insight to how ChatGPT may and may not be of use.
\end{abstract}

\clearpage


\section{Introduction} % (fold)
\label{sec:introduction}

The following paper presents a project conducted as part of the Effective Programming Practices 
course at the University of Bonn. The aim of the project is to produce a framework for 
productivity analysis across various sectors of a country's economy. To achieve this goal, the 
project utilizes web crawaling and web scraping techniques to collect and analyze data from relevant sources. The
 paper will focus on two different measures of productivity: Labour Productivity and Total 
 Factor Productivity. Labour Productivity is a measure of output per hour worked, while Total 
 Factor Productivity is a measure of the overall efficiency of production, taking into account 
 all factors of production, such as labor, capital, and technology. The resulting framework 
 will provide insights into the productivity of different sectors.

The productivity measure standards of the Statistical Bureau of Norway will be used as a 
reference for how productivity is calculated \cite{ssb_norway}. The paper will first present 
changes in productivity over time, followed by levelled graphs for the 10 largest sectors.

Web scraping techniques can be an effective solution to overcome challenges at governmental statistical 
websites, where users might face difficulty and slowness in selecting variables and sectors required for analysis. By 
utilizing web scraping techniques, users can automate the process of navigating statistical websites, 
collecting and extracting the required data from a variety of sources, and ultimately generating useful
insights. Through web scraping, the process of selecting variables and sectors can be automated, 
allowing users to quickly and easily extract the required data, without having to manually search for 
specific data points and remember which variables they use. Furthermore, web scraping can help overcome challenges related to the organization 
of statistical websites, which can often be complex and difficult to navigate. Overall, web scraping provides 
an efficient and reliable method of extracting data from statistical websites, enabling users to overcome 
the challenges associated with traditional data collection methods.

In this paper, as it is a framework, we will only see output in the shape of plots, and the reader is recommended to 
consult the github, for more details.


\begin{figure}[H]

    \centering
    \includegraphics[width=0.85\textwidth]{../bld/python/figures/LP_and_TFP/norway_TFP_plot}

    \caption{Total Factor Productivity in the ten largest sectors in the economy}
    \label{fig:python-predictions}

\end{figure}

\begin{figure}[H]

    \centering
    \includegraphics[width=0.85\textwidth]{../bld/python/figures/LP_and_TFP/denmark_TFP_plot}

    \caption{Total Factor Productivity in the ten largest sectors in the economy}
    \label{fig:python-predictions}

\end{figure}

\begin{figure}[H]

    \centering
    \includegraphics[width=0.85\textwidth]{../bld/python/figures/LP_and_TFP/sweden_TFP_plot}

    \caption{Total Factor Productivity in the ten largest sectors in the economy}
    \label{fig:python-predictions}

\end{figure}


\begin{figure}[H]

    \centering
    \includegraphics[width=0.85\textwidth]{../bld/python/figures/sectors/Telecommunications_compared.png}

    \caption{Total Factor Productivity in the ten largest sectors in the economy}
    \label{fig:python-predictions}

\end{figure}

\begin{figure}[H]

    \centering
    \includegraphics[width=0.85\textwidth]{../bld/python/figures/sectors/Accommodation and food service activities_compared.png}

    \caption{Total Factor Productivity in the financial and insurance activities sector in Scandinavia.}
    \label{fig:python-predictions}

\end{figure}

\begin{figure}[H]

    \centering
    \includegraphics[width=0.85\textwidth]{../bld/python/figures/sectors/Financial and insurance activities_compared.png}

    \caption{Total Factor Productivity in the financial and insurance activities sector in Scandinavia.}
    \label{fig:python-predictions}

\end{figure}

\begin{figure}[H]

    \centering
    \includegraphics[width=0.85\textwidth]{../bld/python/figures/sectors/Tranport and storage_compared.png}

    \caption{Total Factor Productivity in the transport and storage sector in Scandinavia.}
    \label{fig:python-predictions}

\end{figure}

\begin{figure}[H]

    \centering
    \includegraphics[width=0.85\textwidth]{../bld/python/figures/sectors/Construction_compared.png}

    \caption{Total Factor Productivity in the Construction sector in Scandinavia.}
    \label{fig:python-predictions}

\end{figure}

\section{On the use of Chat GPT} % (fold)
\label{sec:On the use of Chat GPT}

I've found that ChatGPT can be helpful for a variety of basic tasks that I encounter 
in my work. The most useful one has been tests. I can use ChatGPT to generate sample test 
cases or to help me come up with different scenarios to test. This saves me time and helps me 
ensure that I'm testing my code.

Another way I've used ChatGPT is to get suggestions for paragraphs that I'm working on. Sometimes 
I get stuck trying to find the right words (especially as english is my second language) or the right way 
to structure a sentence, or I simply find it a bit hard to start writing and ChatGPT can be a useful tool 
to help me brainstorm ideas and get unstuck. I can provide the software with some basic information about 
what I'm trying to say, and it will suggest different ways of expressing it that I might not have thought 
of on my own.

I've also found ChatGPT to be helpful in explaining modules and packages that I'm not familiar with, 
like Selenium. These concepts can be abstract and difficult to understand in the beginning, but ChatGPT 
can help me break them down into more manageable pieces and provide examples that help me understand how they 
work in practice. However, it's important to note that ChatGPT doesn't always provide feasable result, 
and it's still necessary to have a basic understanding of what you're looking for in order to use it correctly,
but also understand what you are ending up using.

ChatGPT can be a valuable tool for problem-solving, as it can help identify errors and generate potential solutions. 
By inputting a description of the problem, it can provide insights that may not have been immediately obvious. It's 
ChatGPT has during this project been an effective aid for troubleshooting and problem-solving, and even though I do 
not know how fast I would have solved the problems without, I imagine it to have saved me a lot of time.

\end{document}