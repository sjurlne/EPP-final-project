\documentclass[11pt, a4paper, leqno]{article}
\usepackage{a4wide}
\usepackage[T1]{fontenc}
\usepackage[utf8]{inputenc}
\usepackage{float, afterpage, rotating, graphicx}
\usepackage{epstopdf}
\usepackage{longtable, booktabs, tabularx}
\usepackage{fancyvrb, moreverb, relsize}
\usepackage{eurosym, calc}
% \usepackage{chngcntr}
\usepackage{amsmath, amssymb, amsfonts, amsthm, bm}
\usepackage{caption}
\usepackage{mdwlist}
\usepackage{xfrac}
\usepackage{setspace}
\usepackage[dvipsnames]{xcolor}
\usepackage{subcaption}
\usepackage{minibox}
% \usepackage{pdf14} % Enable for Manuscriptcentral -- can't handle pdf 1.5
% \usepackage{endfloat} % Enable to move tables / figures to the end. Useful for some
% submissions.

\usepackage[
    natbib=true,
    bibencoding=inputenc,
    bibstyle=authoryear-ibid,
    citestyle=authoryear-comp,
    maxcitenames=3,
    maxbibnames=10,
    useprefix=false,
    sortcites=true,
    backend=biber
]{biblatex}
\AtBeginDocument{\toggletrue{blx@useprefix}}
\AtBeginBibliography{\togglefalse{blx@useprefix}}
\setlength{\bibitemsep}{1.5ex}
\addbibresource{../../paper/refs.bib}

\usepackage[unicode=true]{hyperref}
\hypersetup{
    colorlinks=true,
    linkcolor=black,
    anchorcolor=black,
    citecolor=NavyBlue,
    filecolor=black,
    menucolor=black,
    runcolor=black,
    urlcolor=NavyBlue
}


\widowpenalty=10000
\clubpenalty=10000

\setlength{\parskip}{1ex}
\setlength{\parindent}{0ex}
\setstretch{1.5}


\begin{document}

\title{Final Project: Productivity Analysis Across Sectors}

\author{Sjur Løne Nilsen}

\date{
    {\bf Preliminary -- please do not quote}
    \\[1ex]
    \today
}

\maketitle


\begin{abstract}
    This paper presents a productivity analysis framework for various sectors in a country's 
    economy using web scraping techniques. The analysis focuses on Labour Productivity and 
    Total Factor Productivity. The productivity measure standards of the Statistical Bureau 
    of Norway are used as a reference. The paper presents changes in productivity over time 
    and levelled graphs for the 10 largest sectors. The resulting framework can provide 
    insights into productivity and help policymakers and business leaders optimize productivity 
    across the economy.
\end{abstract}

\clearpage


\section{Introduction} % (fold)
\label{sec:introduction}

The following paper presents a project conducted as part of the Effective Programming Practices 
course at the University of Bonn. The aim of the project is to produce a framework for 
productivity analysis across various sectors of a country's economy. To achieve this goal, the 
project utilizes web scraping techniques to collect and analyze data from relevant sources. The
 paper will focus on two different measures of productivity: Labour Productivity and Total 
 Factor Productivity. Labour Productivity is a measure of output per hour worked, while Total 
 Factor Productivity is a measure of the overall efficiency of production, taking into account 
 all factors of production, such as labor, capital, and technology. The resulting framework 
 will provide insights into the productivity of different sectors.

The productivity measure standards of the Statistical Bureau of Norway will be used as a 
reference for how productivity is calculated \cite{ssb_norway}. The paper will first present 
changes in productivity over time, followed by levelled graphs for the 10 largest sectors.



\begin{figure}[H]

    \centering
    \includegraphics[width=0.85\textwidth]{../bld/python/figures/norway_TFP_plot}

    \caption{\emph{Python:} Total Factor Productivity in the ten largest sectors in the economy}
    \label{fig:python-predictions}

\end{figure}

\end{document}